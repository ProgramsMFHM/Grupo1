\section{Estructuras de datos implementadas}
Dentro de \loopweb se utilizaron en esencia dos estructuras de datos: Tablas hash y Listas enlazadas simples. Ambas tienen sus propias características y usos específicos, y se han diseñado para cumplir con diferentes objetivos propuestos para la aplicación.

%-----------------------------%
\subsection{Tablas Hash}
El uso de esta estructura de datos se debe a su eficiencia a la hora de almacenar información. Esta estructura fue utilizada para almacenar los \textbf{Usuarios}, \textbf{Bandas}, \textbf{Generos} y \textbf{Comentarios} que existen dentro de la ``base de datos'' de \loopweb.

Las tablas hash también son fáciles de crear y eliminar en forma recursiva y los datos que se guardan en ellas pueden ser de cualquier tipo, lo que facilita la gestión de información de los usuarios o bandas quienes no tienen necesariamente un Identificador numérico.

%-----------------------------%
\subsection{Listas Enlazadas Simples}
Esta estructura es el `Engranaje central' de este programa, escogida por su gran flexibilidad en cuanto a espacio.

El principal uso de esta estructura de datos se encuentra en la creación de ``Enlaces a las tablas hash'' anteriores. Esto permite el almacenamiento de información que se encuentra en las tablas sin necesidad de duplicar la información presente en las mismas.

%-----------------------------%
\subsection{Grafos}
El almacenamiento de usuarios es la caracteristica fundamental de \loopweb, esencial para su correcto funcionamiento.

Para ``dar vida'' a esta estructura de datos dentro del programa no se creó una estructura de datos especifica sino que se usó una abstracción de los conceptos de grafos mediante las tablas hash.

Los algoritmos para grafos fueron utilizados para recorrer las conexiones de los usuarios y poder recomendar amigos entre amigos de amigos para ello se usó el algoritmo \textbf{BFS} para recorrer el grafo de usuarios.

%-----------------------------%
\subsection*{¿Qué otras estructuras de datos podrían haber sido implementadas?}
El uso de Listas enlazadas simples permite la facilidad de implementación de todas las funciones propias de \loopweb, sin embargo estructuras como las Listas Doblemente Enlazadas o las Colas podrían dar más eficiencia a ciertas operaciones.
