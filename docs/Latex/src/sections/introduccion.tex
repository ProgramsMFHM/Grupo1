\section{Introducción}
\loopweb es un proyecto diseñado para emular una red social enfocada en la música, desarrollado como una herramienta de aprendizaje práctico en el ámbito de estructuras de datos. Este sistema tiene como objetivo principal implementar y combinar estructuras como grafos, tablas hash y listas enlazadas logrando una solución funcional y adaptativa.

Mediante este desarrollo, se busca consolidar los conocimientos adquiridos a través de la creación de un entorno digital que imita las dinámicas de interacción y conexión propias de las redes sociales modernas.

Entre las funcionalidades implementadas, destacan la gestión de usuarios, gustos y comentarios y la recomendación de conexiones mediante la priorización de contenidos, demostrando la aplicabilidad de conceptos teóricos en un entorno práctico y estructurado.

\section{Objetivos}
El proyecto \loopweb se desarrolló con los siguientes objetivos específicos:

\begin{enumerate}
    \item Diseñar e implementar una simulación funcional de una red social que permita la gestión de perfiles de usuarios, la creación de conexiones entre ellos y la publicación de contenido.
    \item Integrar estructuras de datos complejas, como grafos y tablas hash, en un sistema cohesionado y escalable.
    \item Implementar algoritmos eficientes para optimizar las operaciones del sistema, tales como la recomendación de amigos basándose en afinidades o el ordenamiento de listas para una impresión ordenada por pantalla.
\end{enumerate}
